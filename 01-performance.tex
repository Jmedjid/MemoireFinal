 \chapter{La performance en athlétisme}
\label{part:performance}


Avant de s'intéresser à la problématique posée, nous  allons dans un premier temps définir tous les concepts de ce mémoire et notamment la performance sportive ainsi que ses déterminants.\\

    \section {Qu'est ce que la performance sportive ?}
    
        \subsection{Définition de la performance}
        
            Selon Vladimir Platonov \cite{platonov88}, scientifique dans le domaine des sciences du sport, « la performance sportive exprime les possibilités maximales d'un individu dans une discipline à un moment donné de son développement ». \\
            
            Pour Francis Trilles \cite{trilles02}, c'est « l'aboutissement, le point final (ou intermédiaire) d'une série d'actions appelées préparation sportive. Elle constitue l'objectif d'un long processus d'entraînement. »\\
        
            Nous pouvons également dire que la performance sportive est le résultat d’adaptations motrices aux contraintes et possibilités d'une discipline. 
            Ces contraintes peuvent être d'ordre :
            \begin{itemize}
                \item réglementaires : par exemple au 400m en athlétisme il est interdit de marcher sur la ligne intérieure de son couloir,
                \item événementielles : adaptabilité aux évènements (comme la météo par exemple),
                \item temporelles : par exemple en athlétisme, le temps de réaction au moment du coup de feu,
                \item humaines : ces contraintes peuvent être dépendantes (état de forme du jour) ou indépendantes (chance) de l'athlète.\\
            \end{itemize}
        
            La résolution des problèmes posés par ces contraintes conduit l'athlète à mobiliser des ressources de différentes natures comme ses capacités motrices, techniques, physiques et mentales. Ces ressources peuvent être entretenues et développées grâce à l’entraînement.\\
            
            La performance sportive est donc à la fois \textbf{multifactorielle}, car elle dépend de l'optimisation de chacun des paramètres qui concourent au résultat final et \textbf{systématique}, car les différents facteurs sont interdépendants et unis par des interactions réciproques. Agir sur l'un d'entre eux n'est pas sans conséquence pour les autres.\\
            
        
        \subsection{Éléments constitutifs de la performance}
        
            La performance sportive est constituée de deux parties : visible et invisible.\\
            
            La partie visible est celle à laquelle nous faisons référence communément lorsque nous évoquons la performance. C'est un résultat qui peut être définit par un temps à l'arrivée d'une course, une distance parcourue ou bien un classement. C'est donc un élément quantifiable.\\
            
            La partie invisible pourrait, quant à elle, être définie comme tout ce qui se produit dans notre corps et plus précisément au niveau physiologique lors d'une course. Ce coté de la performance n'est généralement pas analysé par les entraîneurs et athlètes, car elle demande des outils spécifiques et une réelle étude pour trouver des résultats exploitables et intéressants. C'est sur cette partie que portera la présente étude. \\

    
    \section {Les facteurs de la performance en athlétisme}
    
        La performance en athlétisme est un phénomène complexe qui s’articule autour d'un mélange de différents facteurs qui peuvent être indépendants de l'athlète (environnement, chance etc), innés (morphologie, patrimoine génétique) ou encore acquis grâce à l'entraînement (physique, mental etc).\\
        
        Dans les disciplines de courses en athlétisme, nous pouvons séparer les paramètres qui influencent la performance en 6 catégories qui ont elles-mêmes des sous-catégories : 
        \begin{itemize}
            \item physiologiques,
            \item neuromusculaires et relatifs à la condition physique,
            \item psychologiques,
            \item environnement social,
            \item chance,
            \item entraînement invisible.
        \end{itemize}

        \begin{figure}[H]
            \centering
            \includegraphics[scale=0.69]{images/facteursPerf}
            \caption{\label{fig:facteursPerf}Facteurs de la performance en athlétisme}
        \end{figure}
        
        
        \subsection{Facteurs physiologiques}
        
        Les facteurs physiologiques sont les éléments qui permettent de rendre visible la partie invisible de la performance dans le sens où l'étude de ces facteurs permet de comprendre le fonctionnement d'un organisme. Pour un entraîneur, l’étude de la physiologie sportive peut notamment lui permettre d'élaborer des entraînements adaptés au comportement physiologique de l'athlète. Plus précisément, cela permet d'étudier l’adaptation du corps à l'effort au niveau des systèmes nerveux et de l'organisation mécanique et physique de l'athlète, mais aussi au niveau biochimique, cardio-vasculaire et respiratoire.\\
        
        Dans l'expérimentation (\autoref{part:experimentation}), nous nous intéresserons à cette dernière partie et plus particulièrement à l'analyse de la glycémie, de la lactatémie, de l'oxymétrie de pouls et de la fréquence cardiaque.\\
    
    
        \subsubsection{La glycémie}
        \label{glycemie}
        
        La glycémie correspond au taux de glucose (ou taux de sucre) contenu dans le sang sachant que le glucose est la principale source d'énergie de notre organisme. La glycémie oscille en permanence en fonction du moment de la journée, des efforts physiques réalisés et surtout de l'alimentation, mais toute variation importante de celle-ci influe sur le bon fonctionnement du corps. Il est alors important que la glycémie demeure constante et modérée, c'est à dire comprise entre 3,5 mmol/l (63 mg/dl) et 6,1 mmol/l (110 mg/dl) de sang à jeun et inférieure à 7,8 mmol/l (140 mg/dl) deux heures après un repas. \\

        Lorsque le taux de sucre dans le sang est trop faible (inférieur à  63 mg/dl), on parle alors d'\textbf{hypoglycémie} (fatigue, sensation de malaise). Il suffit d'ingérer une dose de glucides (sucre par exemple) pour que la machine reparte dans les minutes qui suivent. Si ce taux est supérieur à 63 mg/dl à jeun ou 140 mg/dl deux heures après un repas, on parle d'\textbf{hyperglycémie}. \\
        
        La glycémie est contrôlée pour maintenir un apport énergétique constant à tous les organes. Elle est régulée par l’action de nombreuses hormones qui peuvent être regroupées en deux groupes. D'une part, les hormones dites « anaboliques »  représentées par l’insuline, qui permet d'abaisser la glycémie, et d’autre part, les hormones antagonistes à l'insuline communément appelées « cataboliques », hormones de stress ou hormones de la contre-régulation à l’hypoglycémie. Celles-ci sont des hormones hyperglycémiantes c'est à dire qu'elles augmentent le taux de glucose dans le sang. Parmi elles, on retrouve le glucagon, l’adrénaline, les glucocorticoïdes et l’hormone de croissance.    \\

        Le taux de glucose dans le sang augmente lors des apports alimentaires pour diminuer petit à petit et retrouver une valeur normale quand les mécanismes de régulation se sont mis en place. Dans ce système régulateur complexe, l'insuline joue un rôle primordial, car elle permet de faire entrer le glucose dans les cellules et en cas d'excès, de diminuer la production de glucose par le foie. \\

        \label{ig}Chaque aliment possède un indice glycémique (IG), qui fait référence à la vitesse d'élévation de la glycémie qu'ils entraînent (taux de sucre dans le sang). Ils sont classés en trois catégories : 
        
        \begin{itemize}
            \item IG élevé : charge glycémique supérieure à 70 (farines blanches et céréales raffinées, sucreries, dattes, etc.);
            \item IG moyen : charge glycémique de 50 à 70 (céréales complètes, fruits secs, etc.);
            \item IG bas : charge glycémique inférieure à 50  (la plupart des fruits et légumes, légumineuses, riz brun, quinoa, etc.)\\
        \end{itemize}
        
        Dans le schéma ci-dessous (fig. \ref{fig:indice_glycemique}), nous pouvons constater que lors de l'ingestion d'un repas composé majoritairement d'aliments à indice glycémique bas (courbe verte), la glycémie reste constante jusqu'à 2 heures après le repas puis diminue progressivement. Par conséquent, la consommation d'aliments à IG bas est préconisée avant un effort, car cela permet une efficacité énergétique plus durable. A l'inverse, nous pouvons remarquer qu'après la consommation d'aliments à indice glycémique élevé (courbe rouge), la glycémie augmente brutalement et atteint un pic 30 minutes après l'ingestion. Elle diminue ensuite tout aussi brutalement et l'athlète se retrouve en état d'hypoglycémie 2 heures après le repas. Ces aliments sont alors déconseillés comme dernier repas avant un effort, mais ils sont utiles juste après. En effet, étant donné qu'ils augmentent instantanément la glycémie, ils facilitent la récupération rapide.
        
        \begin{figure}[H]
            \centering
            \includegraphics[scale=0.6]{images/courbe-index-glycemique.png}
            \caption{\label{fig:indice_glycemique}Mécanisme de l'indice glycémique}
        \end{figure}
        
        
        \subsubsection{La lactatémie}\label{lactatemie}
        
        La lactatémie représente la concentration de lactate dans le sang. Elle est généralement mesurée en exprimé en millimoles de glucose par litre de sang (mmol/l). \\

        Le lactate est un sel majoritairement produit par les muscles, les globules rouges et les cellules cérébrales au cours de la production d'énergie anaérobie (sans oxygène). Il participe à la production énergétique dont les muscles ont besoin pour fonctionner, mais il peut également entraîner des conséquences importantes lorsqu'il est produit en quantité excessive.\\

        Au repos, en temps normal, le sang humain contient une faible concentration de lactate (1 à 2 mmol/l). Toutefois, lors d'un effort intense, les muscles sont soumis à une demande énergétique importante et la succession de contractions y limite la circulation sanguine. Les muscles sont alors moins oxygénés, mais continuent d'utiliser du glucose pour fonctionner.
        Lorsque les cellules musculaires ne sont plus suffisamment oxygénées, voire totalement privées d'oxygène, la consommation de glucose des muscles devient alors supérieure à l'apport d'oxygène dont ils bénéficient et le lactate est sécrété en quantité excessive.
        La mesure de la lactatémie permet ainsi de dépister une mauvaise oxygénation des tissus.\\
      
        En athlétisme ce phénomène opère dans différentes disciplines, mais principalement pendant les courses de 400m, 800m et 1 500m, comme le montre le tableau ci-dessous (fig. \ref{fig:lactatemie}). En effet, c'est au cours de la récupération de ces courses que la concentration de lactate sanguin est la plus importante.
        
         \begin{figure}[H]
            \centering
            \includegraphics[scale=0.71]{images/lactatemie}
            \caption{\label{fig:lactatemie}Valeurs moyennes de lactatémie maximales chez des athlètes de haut niveau selon la discipline.}
        \end{figure}
        
        Ainsi, pendant un effort intense de plus de 20 secondes, la sur-production de lactate et de ce fait l’apparition de la fatigue musculaire est inévitable. Ceci se vérifie quel que soit le niveau de l'athlète.\\
        
        Le graphique ci-dessous (fig. \ref{fig:split_400}) décrit les temps de passage de chacun des 100m composant les records du monde du 400m de Michael Johnson en 1999 et plus récemment de Wayde Van Niekerk en 2016.\\
        
        
        \begin{figure}[H]
            \centering
            \includegraphics[scale=0.85]{images/split-times.png}
            \caption{\label{fig:split_400}Temps intermédiaires pour chaque 100m des records du monde de 400m}
        \end{figure}

        
        Nous pouvons clairement constater que pour les deux coureurs, les temps intermédiaires augmentent significativement dés 20 secondes d'effort. Le dernier 100m (dans la section 300-400m) est même couru plus de deux secondes plus lentement que le deuxième pour Van Niekerk et plus d'une seconde pour Johnson.\\
       
        L'objectif de l'entraînement sera donc de repousser au maximum la rencontre de cet épuisement.\\
        

        \subsubsection{L'oxymétrie de poul}
        \label{oxymetrie}
        
        L'oxymétrie de poul également appelée oxymétrie pulsée représente la quantité d'oxygène qui circule dans le sang et est déterminée par le niveau de saturation de l’hémoglobine en oxygène (SpO2). \\
        
        L’hémoglobine est une protéine contenue dans les globules rouges auxquels elle donne sa couleur. Elle est chargée de transporter l’oxygène dans le sang.\\
        
        Plus précisément, l'oxymétrie représente le pourcentage d’hémoglobine contenant de l’oxygène par rapport à la quantité totale d’hémoglobine dans le sang (hémoglobine oxygénée et hémoglobine non-oxygénée). \\
        
        La saturation pulsée de l’hémoglobine en oxygène est représentée par un pourcentage. Les valeurs normales de SpO2 sont situées entre 95 et 100\%. Si la valeur de la SpO2 est de 98\% cela signifie que chaque globule rouge présent dans le sang est composé de 98\% d’hémoglobine oxygénée et de 2\% d’hémoglobine non-oxygénée. Si la quantité d’oxygène distribuée par le sang aux tissus diminue et que la valeur de SpO2 est inférieure à 95\% cela peut être le signe d’une mauvaise oxygénation du sang aussi appelée hypoxie.\\

        Une bonne oxygénation du sang est importante pour fournir l’énergie nécessaire au fonctionnement des muscles et cela l'est d'autant plus lors d’une activité sportive. En effet, comme nous l'avons vu précédemment, une carence en oxygène entraîne une sur-production de lactate dans le sang et la difficulté de poursuivre un effort.\\
        
        
        \subsubsection{La fréquence cardiaque}
        \label{frequence_cardiaque}
           
        Le coeur est une pompe qui se contracte régulièrement pour assurer la circulation du sang à travers le corps, alimenter les muscles en oxygène et nutriments, mais aussi éliminer les toxines et déchets formés.\\
        
        La fréquence cardiaque correspond au nombre de contractions du coeur ou nombre de battements cardiaques par unité de temps (généralement évaluée sur une minute). Elle peut être mesurée sur plusieurs régions du corps comme le cou, le poignet, l’aine, le doigt ou encore le dessus du pied.
        Sa valeur au repos est généralement comprise entre 50 et 80 battements par minute (bpm) pour un sédentaire et entre 30 et 60 bpm pour un athlète.\\
     
        La fréquence cardiaque est une fonction linéaire du niveau d'effort fournit. Plus l'effort est intense, plus les besoins musculaires augmentent ainsi que les toxines évacuées. Le cœur doit alors battre plus rapidement pour pouvoir répondre à ces besoins. \\
        
        La mesure de la fréquence cardiaque peut être utilisée pour estimer en temps réel l'intensité de l'effort fournit. Sa valeur peut permettre de tester la forme de l'athlète grâce au calcul de la fréquence cardiaque au repos, du VO2max \footnote{Le VO2max ou consommation maximale d'oxygène correspond au volume maximal d'oxygène prélevé au niveau des poumons et consommé par les muscles par unité de temps.} et de la fréquence cardiaque maximale (FCM), mais aussi de gérer l'effort fourni. Après une course, l'exploitation de cette donnée peut permettre d'apprécier la capacité de récupération de l'athlète, mais également d'optimiser et de personnaliser les futurs entraînements. \\
    
        
        \subsection{Facteurs neuromusculaires et relatifs à la condition physique}
        
        Les facteurs neuromusculaires concernent la relation entre le système nerveux et le système musculo-squelettique. Ils permettent aux muscles de se contracter en fonction des différentes informations reçues.\\
        
        Ces facteurs sont généralement les plus complets et représentent les aspects de la performance qui occupent le plus grand degré de concentration et de temps de préparation. En effet, dans de nombreux sports, peu importe la façon dont l'athlète se consacre à l'entraînement, s'il n'est pas physiquement équipé pour concourir, la performance ne s'améliorera pas.\\
        
        La composante neuromusculaire de la performance sportive est subdivisée en ses propres éléments discrets qui doivent chacun faire l'objet d'entraînements spécifiques.\\
        
        Les entraînements principaux concernent :
            \begin{itemize}
                \item les filières énergétiques,
                \item la souplesse,
                \item la technique de course.\\
            \end{itemize}
        
        Ces entraînements sont influencés par l'anatomie de l'athlète et par la distribution de ses fibres. En effet, les muscles sont composés essentiellement de fibres musculaires.\\
      
        On compte principalement deux types de fibres musculaires :
        \begin{itemize}
            \item \textbf{les fibres rouges} (fibres lentes de type I) : peu de vitesse et peu de force, mais grande forte capacité de résistance à l’effort (endurance),
            \item \textbf{les fibres blanches} (fibres rapides de type II) : vitesse de contraction rapide et beaucoup de force, mais faible résistance à l’effort et incapables de se contracter longtemps (peu d'endurance).\\
        \end{itemize}
        
        C'est en fonction du type d'effort pratiqué que l'on utilisera l'une plutôt que l'autre (ou les deux) (fig. \ref{fig:fibres}).
        
        \begin{figure}[H]
            \centering
            \includegraphics[scale=0.7]{images/fibres.jpg}
            \caption{\label{fig:fibres}Fibres sollicitées lors de différents types de courses}
        \end{figure}
        
        A la naissance, la composition des muscles est programmée génétiquement pour 50\%. Cela signifie que la moitié de nos fibres musculaires sera non modifiable et l’autre moitié sera transformable par l’entraînement.\\

        C'est donc en fonction des prédispositions génétiques de l'athlète, de ses aptitudes physiologiques et musculaires ainsi que de sa discipline que l'athlète va produire l'énergie nécessaire à l'accomplissement de son effort.\\
      
        Il existe trois modes de production d'énergie : 
        \begin{enumerate}
            \item la filière anaérobie alactique (sans oxygène et sans production de lactates),
            \item la filière anaérobie lactique (sans oxygène et avec production de lactates),
            \item la filière aérobie (avec oxygène).\\
        \end{enumerate}
        
        Chacune d’elle est caractérisée par une capacité (quantité totale d'énergie potentielle) et une puissance (quantité d'énergie délivrable par unité de temps). Par analogie avec un réservoir muni d'un robinet, la capacité correspond au volume du réservoir et la puissance au débit du robinet. 
        Bien que fonctionnant en parallèle, ces filières possèdent des durées de fonctionnement spécifiques qui les prédisposent à trois types d'efforts différents.\\
        
        \begin{description}
            \item  [La filière anaérobie alactique]
        est assurée par les fibres à contraction rapide des muscles. 
        Cette filière s’enclenche dès les premières secondes de l’exercice et est utilisée essentiellement dans les efforts réalisés au maximum d’intensité en un temps inférieur à 20 secondes. Toutefois, sollicitée à son maximum d’intensité, la filière anaérobie alactique est épuisée au bout de 7 secondes. En athlétisme les disciplines concernées par ce métabolisme énergétique sont les sprints explosifs tels que le 100 mètres.  Pour les courses plus longues (200 m, 400 m, 800 m), et les courses de haies (110 m haies et 400 m haies), le métabolisme anaérobie alactique est mis en jeu lors des 7 à 20 premières secondes, puis c'est la filière anaérobie lactique qui prend le relais.\\
        
        \item [La filière anaérobie lactique]
        se met en place lorsqu'il n’y a plus assez d’arrivée d’oxygène pour répondre à l’effort supérieur. Le corps fait alors appel à cette filière, qui consiste en une aide au fonctionnement du muscle. Toutefois, cette aide n’est pas sans conséquences. La filière anaérobie lactique utilise du glycogène, un sucre complexe présent à l'état de réserve au niveau du muscle, qui à la suite de réactions chimiques complexes, se transforme en lactate. L'accumulation de ces lactates dans le sang entraîne un apport d’acidité dans l’organisme qui perturbe la contraction du muscle.
        Résultat, la fatigue musculaire se fait sentir et la faculté de poursuivre son effort se réduit. \\
        
        Ce métabolisme s’enclenche dès les premières secondes de l’exercice, mais avec une intensité inférieure à celle du processus anaérobie alactique.
        Il assure des efforts de puissance élevée et de durée moyenne, de 20 à 45 secondes. C'est ce processus qui prédomine dans les courses de résistance comme le 400m ou le 400m haies.\\
        
        Le but de l’entraînement lactique est donc de produire beaucoup de lactate pour habituer l’organisme à ressentir la douleur, à la supporter pour ensuite repousser au maximum les limites de la tolérance. C'est ce qui permettra de retarder l'apparition de la fatigue musculaire.\\
        
        
        \item [La filière aérobie]
         représente les gammes d’intensité de travail d’endurance et est mis en place lors d'un effort long et modéré.
        Elle s'enclenche également dés le début de l'effort, mais c'est au bout de 3 minutes qu'elle devient la seule source d'énergie étant donnée que les deux autres ont été épuisées.
        La filière aérobie correspond à l’aptitude de l’organisme à capturer (respiration), transporter (globules rouges et débit cardiaque) et utiliser (efficacité oxydative des cellules) l’oxygène pour transformer l’énergie.
         
        C'est une qualité essentielle au succès quelle que soit la spécialité de l'athlète, car elle permet de régénérer les sources précédentes.\\
         
        Contrairement à ce que l'on pourrait penser, même dans les disciplines de courte durée et à haute intensité telles que le sprint par exemple, l'aérobie tient un rôle très important. Elle permet une meilleure oxygénation des muscles et participe à l’élimination des déchets de la contraction musculaire, ce qui favorise une récupération plus rapide et plus efficace après un entraînement ou une compétition. \\
        
        Dans les disciplines de fond, l'aérobie de l'athlète, c'est à dire sa capacité à consommer et traiter l'oxygène est primordiale. En effet, un des axes de travail du processus aérobie est de maintenir une intensité élevée le plus longtemps possible tout en dépensant le moins d'énergie possible.\\
        
        Cependant, ce processus n'est pas sans limite. 
        Lorsque l’athlète se rapproche du seuil critique du processus aérobie c'est à dire des limites pour lesquelles tout l’oxygène disponible au niveau musculaire est utilisé, on dit qu'il atteint son VO2max. \\
        
        Néanmoins, un athlète ayant atteint son VO2max peut encore augmenter l’intensité de son exercice. Ne disposant plus de réserves d’oxygène, il fera à nouveau appel à ses processus anaérobies, ce qui va provoquer une augmentation importante de la lactatémie. C’est le cas par exemple lors du sprint final d’un 1500m.\\
        
        \end{description}
     
         Dans le schéma ci-dessous (fig. \ref{fig:filieres-energetiques}), nous pouvons clairement voir que les trois sources ne sont pas indépendantes. Elles fonctionnent en parallèle, à des degrés divers, ce qui crée une illusion de fonctionnement en série. Pour chaque source, la surface comprise entre la courbe et l'axe du temps représente la capacité (l'endurance).\\
         
         \begin{figure}[H]
            \centering
            \includegraphics[scale=0.75]{images/aerobie-anaerobie-alactique2.png}
            \caption{\label{fig:filieres-energetiques}Les filières énergétiques}
         \end{figure} 
         
         
         La filière anaérobie alactique est celle qui produit le plus de puissance, mais qui a la plus faible capacité, c'est à dire qu'elle ne fournit pas d'énergie aux muscles longtemps : elle ne peut pas prolonger un effort à 100\% de son VO2max plus de 7 secondes. Au-delà de cette période, les muscles doivent utiliser d’autres procédés pour continuer la couverture énergétique. C'est le métabolisme anaérobie lactique qui prend le relais. Cette filière a une capacité et une puissance moyenne. Elle est active jusqu'à 3 minutes d'effort, mais produit une puissance élevée jusqu'à 45 secondes d'effort. Ensuite, la réserve d'énergie s'affaiblie et c'est la filière aérobie qui prend le relais. Ce métabolisme a une endurance très supérieure à celles des processus anaérobies, mais une puissance inférieure à celle assurée par ces derniers. A partir d'environ 3 minutes d'effort elle devient la seule source d'énergie et elle délivre sa puissance maximale vers 3 minutes 30 d'effort.\\
        
        La distribution des fibres à travers les muscles est donc régulée par la génétique, mais cette distribution n'est pas irréversible. Le but de l'entraînement est d'entraîner les fibres dans les filières énergétiques les plus importantes selon la spécialité de l'athlète pour maximiser leurs effets afin d'être le plus efficace possible. \\
        
        En plus des entraînements pour augmenter ses capacités énergétiques, l'athlète travaille aussi sa souplesse.\\
                

        \subsubsection{La souplesse}
        La souplesse désigne la qualité physique permettant d’accomplir des mouvements corporels avec la plus grande amplitude (articulaire et musculaire) et aisance possibles. que ce soit d’une manière active (en mouvement dynamique) ou passive (sans mouvement dynamique).\\
    
         On peut distinguer différents types de souplesse :
         \begin{itemize}
             \item la souplesse générale : mobilise les systèmes musculaires et articulaires pour faire en sorte d’apporter une aisance gestuelle;
             \item la souplesse spécifique : se rapporte à un muscle ou une articulation spécifique (e.g., franchir des haies nécessite une grande souplesse au niveau des ischios jambiers);
             \item la souplesse statique et dynamique : résulte de la présence ou non d’un mouvement;
             \item la souplesse active et passive : concerne la souplesse statique, leur différence réside dans la présence ou non d’une contraction musculaire;\\
         \end{itemize}

        En athlétisme, la souplesse est une qualité essentielle. Plus l'amplitude de mouvement présente dans les articulations d'un athlète est grande, meilleure sera l’efficacité du geste et la capacité de mouvement dynamique.\\

        La souplesse se travaille grâce à des étirements.\\
        
        Il existe deux types d'étirements: 
        \begin{itemize}
            \item les étirements avant séance :  pour réveiller les muscles, mais aussi augmenter et faciliter les amplitudes articulaires et musculaires (allongement des tissus et des muscles) qui seront utilisés lors de l'effort;
            \item les étirements après séance : pour permettre aux muscles de récupérer, mais aussi de prévenir les blessures.
            Il s’agit ici de relâcher et de décontracter les muscles pour retrouver un degré de tonus musculaire presque similaire au début de l’effort.\\
        \end{itemize}
                

        \subsubsection{La technique}
        
        En athlétisme, la maîtrise technique est un aspect indispensable pour le haut niveau.\\
        
        Cela englobe :
        \begin{itemize}
            \item la qualité des gestes (placements du corps et des articulations),
            \item la précision technique (pose de pied sous le bassin par exemple),
            \item la vitesse d’exécution (qui dépend aussi de la qualité et précision des gestes),
            \item la capacité de coordination.\\
        \end{itemize}
        
        Généralement la technique n'est pas innée et se travaille grâce à la répétition et à des exercices spécifiques. Cependant, il n'y a pas une seule technique efficace et universelle, mais plutôt une technique adaptée à chaque athlète en fonction du type de corps et des capacités musculaires de chacun.\\
    
        Ces différents facteurs ne sont pas les seuls déterminants de la performance en athlétisme. Nous allons maintenant nous intéresser à d'autres facteurs tout aussi importants.\\
        
                
        \subsection{Facteur psychologique}

            Le contrôle mental et les aspects psychologiques associés à la performance sportive sont des éléments déterminants qui se reflètent dans le résultat final de l'effort de l'athlète.\\
            
            En athlétisme, on dit que le mental joue pour 70\% de la performance et le physique pour seulement 30\%. C'est entre autres lui qui sépare les athlètes qui réussissent de ceux simplement talentueux. \\
            
            Toutefois, les éléments mentaux du sport sont à bien des égards les plus difficiles à maîtriser. Ils requièrent habituellement un haut niveau de maturité et d'expérience athlétique pour aboutir à des résultats et avoir la capacité d'être performant le jour J.\\
            
            Il n'est en effet pas rare de trouver des athlètes extrêmement doué physiquement, mais qui sont incapables de gérer la pression et de maîtriser leurs émotions pendant une compétition. Nous pouvons prendre l'exemple de Marie-José Perec, athlète surdouée et multiple médaillée olympique dans les épreuves de 200 et 400m, qui avait précipitamment quitté les Jeux Olympiques la veille de sa course sans explications.\\
            
            Les capacités cognitives de l'athlète (intelligence, mémoire, langage) peuvent expliquer certains comportements, car elles sont directement impliquées dans le processus mental. Elles concourent à l’appréhension et au traitement des informations ainsi qu'à l’analyse des situations. Le fait d'avoir une grande puissance analytique permet à athlète de prendre le contrôle de ses émotions et ainsi rester calme en situation de stress ou sous pression.\\
         
            La maîtrise émotionnelle n'est pas le seul point important de la force mentale d'un athlète. En effet, la capacité du sportif à avoir confiance en soi ainsi que son aptitude à être persévérant pour poursuivre son objectif jusqu’au bout, quelles que soient les difficultés rencontrées, sont également des qualités primordiales.\\
            
            Enfin, une des clés essentielles du succès est d'avoir un désir de réussite hors du commun. C'est cette envie qui permettra à l'athlète d'accepter la souffrance tout en se faisant plaisir, de s'auto-motiver et de pouvoir se surpasser tant en compétition qu'à l'entraînement.\\
            
                
        \subsection{Facteur social}
            
            L'athlète est un compétiteur, mais aussi et avant tout, un être humain évoluant dans un environnement et un contexte social au quotidien.\\
            
            Nous pouvons définir l’environnement comme étant un ensemble de conditions et de processus, englobant le système « entraînement-compétition » et plus globalement, la vie de l'athlète. \\
            
            Cet environnement peut être : 
            \begin{itemize}
                \item matériel : piste par exemple;
                \item physique : conditions climatiques, altitude;
                \item sportif : logique de formation / compétition, catégories et niveaux des athlètes;
                \item fédéral : calendrier des compétitions;
                \item relationnel : dynamique au sein du groupe; d’entraînement, situation familiale, affective ou professionnelle, relation entraîneur/entraîné.\\
            \end{itemize}

            Les pensées, sentiments et comportements de l'athlète sont influencés par son environnement et notamment par la présence réelle, imaginaire ou implicite des personnes qui gravitent autour de lui. La relation entraîneur/entraîné est un des éléments, si ce n'est l'élément le plus important dans l'environnement d'un sportif. Elle est généralement basée sur une entente entre une personne qui veut atteindre des objectifs et un guide qui peut l’y amener. Ce lien repose donc sur un désir de performance et peut être de différentes natures : amical, mentor/athlète, fraternel, etc. Quelle que soit la nature de cette relation, l'athlète et l'entraîneur passent environ 4h à 5h par jour ensemble dans le but d'atteindre un objectif commun. La qualité de la communication et la confiance entre les deux individus sont donc des éléments indispensables et déterminants dans leur quête de la réussite.\\
            
            Dans cette relation, le rôle de l'entraîneur est primordial. C'est lui qui va organiser les entraînements afin de favoriser le développement des capacités de l'athlète et l'amener à l'accomplissement de ses objectifs. C'est donc un gestionnaire, mais pas seulement. L'entraîneur est la personne qui va inspirer confiance à l'athlète, le soutenir dans les moments de doutes et qui va l'aider à utiliser des ressources connues ou inconnues pour solutionner un problème. C'est aussi celui qui va aider l'athlète dans son développement personnel afin d'augmenter son bien-être au quotidien.\\
         
         
            L'environnement social de l'athlète module et influence donc l'organisation et la gestion des entraînements/compétitions, mais aussi leur réalisation.
            Cependant, les facteurs environnementaux sont rarement sous le contrôle personnel de l'athlète. La capacité de l'athlète à s'adapter à des facteurs environnementaux différents et inattendus est souvent déterminante pour la réussite. \\
            
            
        \subsection{Facteur chance}    
            
            Le facteur chance est un facteur qui peut être controversé, mais qui existe néanmoins. Il conditionne les éléments extérieurs et non contrôlables autour de la performance.\\
            
            Le facteur chance englobe :
            \begin{itemize}
                \item le nombre et le niveau des adversaires;
                \item les éventuelles blessures et abandons des adversaires;
                \item les erreurs de jugement (athlète disqualifié par erreur par exemple);
                \item la configuration des séries et des couloirs. Par exemple, en athlétisme et notamment aux 400m/400m haies, le couloir 1 est le pire couloir en terme de rapidité du fait de sa proximité avec le centre du stade et par conséquent, de ses virages serrés. Cependant, lors des séries, les couloirs sont tirés au  hasard et il est possible pour tous les athlètes de l'avoir. Un autre exemple ou la chance entre en jeu est lors de la configuration des couloirs par rapport aux concurrents : un athlète qui est au couloir 5 aura un avantage par rapport à celui qui est au couloir 6, car il bénéficiera du visuel sur lui. Il pourra ainsi le contrôler plus facilement et éviter des surprises;
                \item les conditions météorologiques. Ces conditions sont un facteur qui sera le même pour tous les concurrents. Il faudra alors avoir une meilleure capacité d'adaptation que les autres.\\
            \end{itemize}
            
            Pour un athlète cherchant à maximiser sa performance, il ne doit pas seulement exercer un contrôle mental pour éviter d'être dérangé par ces conditions, mais il doit également examiner les moyens de faire fonctionner les conditions dans le positif.\\
            
             
             
        \subsection{L'entraînement invisible}
         
            Chaque séance d’entraînement, chaque compétition provoque une dépense énergétique proportionnelle à l’intensité et la durée de la sollicitation musculaire. \\
             
            L'entraînement invisible correspond à optimiser tout ce qui tourne autour de l'entraînement physique pour rentabiliser au maximum ses effets. Il s’agit des domaines que l’on pourrait classer dans l’hygiène de vie et qui permettent d'optimiser la récupération. \\
            
            Concrètement, cela consiste à se préparer au mieux à l’effort, à optimiser son alimentation pour restaurer les réserves énergétiques et les tissus utilisés ainsi qu'à s'entretenir par des soins divers, tout cela pour éviter le surentraînement et prévenir l'apparition de blessures.\\
            
            Le surentraînement est un phénomène qui intervient lorsque la durée de récupération entre les séances n’est pas respectée ou mal gérée. Cela engendre une accumulation de la fatigue et une régression progressive des capacités physiques de l'athlète. Il est alors primordial de trouver le bon équilibre entre activité physique (entraînement, compétition) et récupération (repos, alimentation, soins, etc.). Cette phase de récupération va permettre à l'ensemble des systèmes sollicités pendant l’effort de se restructurer et de régénérer les réserves énergétiques.\\
              
            La récupération est donc l'une des clés de la performance.\\
    
            Les paramètres d'une bonne récupération sont : 
             \begin{itemize}
                 \item le sommeil,
                 \item les soins,
                 \item l'alimentation et l'hydratation.
                
             \end{itemize}
             
            Beaucoup d'athlètes négligent partiellement ou totalement l'entraînement invisible, pourtant il est tout aussi important que l'entraînement physique. \\
            
    
                \subsubsection{Le sommeil}
        
                    Il est reconnu qu’un manque de sommeil ou un sommeil perturbé peut nuire au fonctionnement mental et physique, affaiblir le système immunitaire et affecter les autres processus de rétablissement importants pour les athlètes.\\
                    
                    Un sommeil de qualité est alors indispensable pour un athlète. C’est pendant le sommeil qu’il y a une grande circulation des hormones de croissance et par conséquent une régénération des tissus cellulaires endommagés par l’entraînement. Il permet ainsi à l’organisme de récupérer des efforts qu’il a fourni. \\
                    
                    Une bonne nuit de sommeil se caractérise par :
                    \begin{description}
                        \item [Le besoin en sommeil.] La quantité de sommeil nécessaire diffère selon chacun, mais un athlète ne devrait pas dormir moins de 8 heures par nuit. Celle-ci peut être complétée si nécessaire par des siestes.
                     
                        \item [La qualité du sommeil.] Plusieurs athlètes dorment suffisamment d’heures,
                        mais leur sommeil est de mauvaise qualité. Le sommeil n'est alors pas aussi réparateur.
                        \item [Le moment du sommeil.] Chaque athlète présente un chronotype particulier (préférence de l’individu à réaliser une activité à certaines périodes de la journée plutôt qu’à d’autres) : certains sont matinaux et d'autres non. Les horaires d'entraînement doivent donc être synchronisés avec le rythme de l'athlète pour ne pas perturber son besoin en sommeil.
                        \item [Les habitudes de sommeil.] Il est important d'adopter des habitudes de sommeil pour favoriser la récupération. Par exemple, instaurer des heures de coucher avant minuit puisque celles-ci sont les plus «récupératrices».\\
                    \end{description}
                
        
                 
                \subsubsection{Les soins}
   
                Pour un athlète, le corps est l'élément indispensable à la réalisation de l’objectif qu’il s’est fixé. Cependant, les entraînements quotidiens malmènent les muscles, tendons, articulations, etc.  Il faut donc prendre soin de son corps et faire en sorte qu'il soit dans le meilleure état possible pour affronter les lourdes charges d'entraînements. Il faut également que l'organisme consacre toute son énergie à régénérer ses réserves au lieu de lutter contre les inflammations et les blessures.\\
                
                Pour cela, le sportif doit réaliser des soins qui permettront de prévenir les blessures ou de les traiter, mais aussi à récupérer (soins dentaires, podologie, ostéopathie, kinésithérapie, cryothérapie, étirements, massages, etc.).\\
                
            
                \subsubsection{L'alimentation et l’hydratation}
                
                    Le corps humain a besoin d’apports externes pour faire fonctionner l’organisme et créer de l’énergie. Ces apports proviennent de l'alimentation et de l'hydratation quotidienne et sont d'autant plus importants chez un athlète du fait de ses hautes dépenses énergétiques.
                    Ils doivent comblés quantitativement et qualitativement les dépenses énergétiques de l'athlète afin de reconstituer les stocks d'énergie perdus.\\
                    
                    Outre la quantité et la qualité des apports, la répartition quotidienne des prises alimentaires doit être respectée : l’organisme doit être suffisamment alimenté avant l'effort pour pouvoir être performant, mais aussi après pour restaurer ses réserves énergétiques et se réhydrater.\\
                    
                    Après un effort intense, l'athlète dispose d'une «fenêtre métabolique» qui correspond à un intervalle de temps pendant lequel les aliments ingérés sont assimilés plus rapidement. C'est le moment favorable pour rétablir le stock d'énergie, c'est à dire le stock de glycogène musculaire.\\
                    
                    Ce phénomène prédomine durant la première heure après l’effort et diminue rapidement. Si l'apport glucidique n'est pas absorbé par l'athlète dans les 4 heures suivant l'effort, il lui faudra 3 jours pour récupérer à 100\%.
                    
                    \begin{figure}[H]
                        \centering \includegraphics[scale=1.2]{images/fenetre-metabolique.jpg}
                        \caption{\label{fig:fenetre_metaboliques}La fenêtre métabolique}
                    \end{figure}
                    
                    
                    De la même façon, il est important d’assurer un bon équilibre hydrique en s'hydratant régulièrement, mais aussi d'adapter son apport en eau en fonction des pratiques en buvant avant, pendant et après l’effort. En effet, la transpiration et la sudation entraînent une perte d'eau supplémentaire. Lorsque la perte d’eau correspond à 2\% du poids de corps, la performance baisse de 20\%, si cette perte correspond à 4\% du poids de corps, à 18° de température ambiante la performance baisse de 40\%!  De plus, l'eau favorisera la restauration du stock minéral perdu et contribuera à diminuer l’acidité musculaire dû à l'effort.
    
              \vspace{20pt} 
             

        La performance est donc un système complet et organisé qui comprend une multitude de facteurs qui en sont les fondements. Ils se révèlent plus ou moins importants selon la situation, les caractéristiques de l’athlète et le contexte, mais ils doivent être connus et intégrés dans le processus d’entraînement afin de maximiser les performances de l'athlète et d'éviter fatigue et surentraînement. 


    
    

%%% Local Variables: 
%%% mode: latex
%%% TeX-master: "memoire"
%%% End:





